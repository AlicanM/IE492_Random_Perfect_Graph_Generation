\documentclass[12pt, oneside, a4paper]{article}
%== Preamble ==%
\usepackage[dvipsnames]{xcolor}
\usepackage{adjustbox}
\usepackage{hyperref}
\usepackage{graphicx}
\usepackage{svg}
\usepackage{csquotes}
\usepackage[turkish,shorthands=off,english]{babel}
\usepackage[backend=biber, sorting=nty]{biblatex}
\usepackage{tikz}
\usepackage[algo2e]{algorithm2e}
\usepackage{algorithm}
\usepackage{tabularx}
\usepackage{diagbox}
\usepackage{enumitem}
\usepackage{booktabs}
\usepackage{makecell}
\usepackage{cellspace}
\usepackage{multirow}
\usetikzlibrary{graphs,graphdrawing, calc, positioning, shapes, arrows.meta}
\usegdlibrary {trees, force, circular}

\bibliography{proposal_references}

% Motivation
% Algorithmic Graph Theory and Perfect Graphs (Second Edition) - Golumbic, Martin Charles

% Graph Classes, A Survey - Brandstädt Andreas, Le Van Bang, Spinrad Jeremy P.

\newcommand{\PerfectGen}{\texttt{PerfectGen}}

% For sorting the references alphabetically
\DeclareSortingNamekeyTemplate{
  \keypart{
    \namepart{given}
  }
  \keypart{
    \namepart{prefix}
  }
  \keypart{
    \namepart{family}
  }
  \keypart{
    \namepart{suffix}
  }
}

% Hyperlink setup
\hypersetup{
    colorlinks=false,
    pdftitle={IE 492 Final Project Report: Generating Random Perfect Graphs} ,
    pdfauthor = {Ali Can Milani, Aral Dörtoğul, Fatih Mehmet Yılmaz},
    bookmarksopen = true,
    colorlinks = false,
    pdfpagemode=FullScreen,
    hidelinks,
    breaklinks
    }

\renewcommand{\contentsname}{Table of Contents}

%== Document ==%

\begin{document}

\pagenumbering{roman}

\begin{titlepage}
    \begin{center}
            
        \LARGE
        IE 492: Project
            
        \vspace{1cm}
        \includegraphics[width=0.5\textwidth]{images/boun_logo.png}
        
        Final Project Report\\[0.3cm]
        \huge
        Generating Random Perfect Graphs
        
        \vspace{0.5cm}
        \LARGE
        Advisor: Tınaz Ekim

        \vfill
        
        \begin{table}[h]
        \centering
        \Large
        \begin{tabular}{p{6.5cm}r}
        Ali Can Milani & 2018402171 \\
        Aral Dörtoğul & 2018402108 \\
        Fatih Mehmet Yılmaz & 2017402066
        \end{tabular}
        \end{table}
            
        \vfill
            
        \large
        Industrial Engineering\\
        Boğaziçi University
        
        \vspace{0.8cm}

        \today
            
    \end{center}
\end{titlepage}

\newpage

\begin{abstract}
    This project focuses on the generation of random perfect graphs, specifically exploring their practicality for testing algorithms designed for perfect graphs. Different algorithms are developed and compared through a detailed analysis of their performance, namely the ``heuristic repair method'' and the ``combinatorial approach''. The study aims to generate graphs that have the properties of perfect graphs and evaluate the efficiency and accuracy of each algorithm. The outcomes of this research contribute to algorithm testing and performance evaluation, aiding researchers in selecting suitable methods for their requirements and advancing the field of perfect graph theory. In result, the combinatorial algorithm works faster, but it is limited by a number of operations; whereas, the heuristic repair approach has the possibility to generate every perfect graph with a positive probability even though it takes much more time compared to the combinatorial method.
\end{abstract}

\begin{otherlanguage}{turkish}

    \begin{abstract}
        Bu proje, özellikle ``mükemmel'' çizgeler için tasarlanmış algoritmaların test edilmesi açısından rastgele mükemmel çizgelerin üretimine odaklanmaktadır. ``Sezgisel onarım yöntemi'' ve ``kombinatoryal yaklaşım'' gibi farklı algoritmalar geliştirilmiş ve performanslarının ayrıntılı bir analizi yoluyla karşılaştırılmıştır. Çalışmanın amacı, mükemmel çizgelerin özelliklerine sahip çizgelerin üretilmesini sağlamak ve her bir algoritmanın etkinlik ve doğruluklarını değerlendirmektir. Araştırmanın sonuçları, algoritma testi ve performans değerlendirmesine katkıda bulunarak araştırmacıların ihtiyaçlarına uygun yöntemleri seçmelerine yardımcı olacak ve mükemmel çizge kuramı alanının gelişmesine katkı sağlayacaktır. Sonuç olarak, kombinatoryal algoritma daha hızlı ancak sayılı sayıda işlemle sınırlı kalmaktadır; sezgisel yaklaşım ise, kombinatoryal yönteme göre çok daha fazla zaman almasına rağmen her mükemmel çizgeyi pozitif bir olasılıkla oluşturma olanağına sahiptir.
    \end{abstract}

\end{otherlanguage}

\newpage

\section{Introduction}

\subsection{Summary}

The study of ``perfect graphs'' has been an intriguing and challenging area in graph theory, with numerous applications in fields such as optimization, network design, and scheduling. These graphs possess remarkable structural properties and have been extensively studied over the years.

The problem addressed in this study is the generation of random perfect graphs, which plays a crucial role in understanding their properties and developing algorithms for practical applications. The aim of this study is to investigate and analyze various algorithms for generating random perfect graphs. In this context, the primary objective is to develop efficient and effective methods for generating random perfect graphs that accurately capture the key properties of these graphs. Additionally, we seek to compare the algorithms by exploring the statistical properties and structural characteristics of the generated graphs, providing insights into the behavior and distribution of random perfect graphs.

The findings of this research will not only deepen our understanding of perfect graphs but also have practical implications for algorithm design and problem-solving on perfect graphs in real-world applications.

\subsection{Short Overview of Identification, Analysis and Solution Methodologies}


Firstly, the problem is clearly defined and the study's objectives are outlined, considering the necessary requirements and limitations. Fundamental concepts and definitions in perfect graph theory, such as perfectness, cliques, and related properties are also reviewed. Then, existing methods for generating perfect graphs in the literature such as Erdős and Rényi's approach~\cite{erdos} are examined. Building upon this foundation, novel algorithms for generating random perfect graphs like ``heuristic repair algorithm'' and ``combinatorial algorithm'' are proposed and their performance in terms of efficiency and quality of the generated graphs are analyzed. Further, an extensive experimental evaluation is conducted to validate the proposed algorithms and assess the statistical properties of the generated random perfect graphs. The generated graphs are compared against known benchmarks and their performance are assessed using various graph metrics and properties to provide valuable insights into the effectiveness and limitations of the proposed algorithms. Finally, the work is concluded by summarizing the key findings and discussing the potential future improvements to the algorithms.

\newpage

\tableofcontents

\newpage

\pagenumbering{arabic}

\section{Problem Definition, Requirements, and Limitations}

\subsection{The Problem}

Graph theory is a mathematical field concerned with the study of graphs, which are mathematical structures consisting of ``vertices'' or nodes connected by ``edges'' or arcs, as shown in Figure~\ref{fig:graphExample}. Graph theory has numerous applications in various fields such as computer science, engineering, physics, social sciences, and operations research.

\begin{figure}[h]
    \centering
    \begin{tikzpicture}
    [
        Text/.style={fill=red!10, rounded corners=5pt},
        Arrow/.style={dashed, <-, thick}
    ]
        \graph
        [spring layout,
        nodes={draw, thick, circle, minimum size=8mm},
        node distance=15mm,
        horizontal=f to a]
        {
            a -- { b, c, e -- {f, g, h} };
            { h, g } -- a;
        };
        \draw[Arrow] (c) to[out=0,in=180] +(13:25mm) node [Text] {Vertex};
        \draw[Arrow] ($ (a)!.5!(b) $) to[out=-45, in=180] ++(8:30mm) node [Text] {Edge};
    \end{tikzpicture}
    \caption{A Graph}
    \label{fig:graphExample}
\end{figure}


Graphs are used to model many real-world situations, such as social networks, transportation networks, electrical circuits, and communication networks. For example, in social networks, graphs are used to represent people as nodes and relationships between them as edges. In transportation networks, different cities are modeled as edges and the cities themselves as nodes. 
The practical applications of solving algorithmic problems on special graph classes span across various fields, including VLSI circuit design, scheduling, resource allocation, DNA mapping, artificial intelligence's temporal reasoning, and pavement deterioration analysis. Additionally, perfect graph classes have contributed significantly to the development of profound theoretical findings~\cite{golumbic}. Exploring these graph classes is a starting point for researchers to pursue various fascinating research directions. Therefore, while real-life problems are modeled with graphs, the resulting graphs have certain structural features coming from the nature of the problem.

Many algorithms have been proposed in the literature to solve real-life problems modeled by these graphs. A significant amount of these algorithms perform with exponential time complexity due to the NP-hardness of the problems. However, algorithms designed for specific graph classes (such as \textit{perfect graphs}) can solve these problems much faster by utilizing that graph class' special properties in the algorithm. Therefore, for some graph classes, these problems can be solved faster than the generalized algorithms with a polynomial time complexity.

In order to verify and evaluate the performance of the algorithms for special graph classes, they need to be tested for the whole set of graphs of its specific class. However, examples of graphs belonging to the related graph classes are often not available to measure the performance of these algorithms empirically. Tests can be conducted on graphs that are known to belong to the related graph class, but in some cases it is not known whether these graphs are randomly selected from that graph class and reflect all its diversity. In such cases, tests carried out with those graphs can lead to the incorrect assessment of the algorithms. Therefore, being able to generate random graphs of related classes as unbiased as possible is crucial for testing the algorithms implemented specifically for the related graph class.

In this project, \textbf{random perfect graph generation} is studied, because \textit{perfect graphs} and their sub-classes such as \textit{chordal graphs} and \textit{interval graphs} are at the forefront of the graph classes that stand out with their structural features coming from the applications. Generating random perfect graphs can be a practical solution to test the algorithms that work only for perfect graphs and analyze their performance. In this study different algorithms are designed to generate random perfect graphs and are compared with a detailed analysis.

\subsection{Understanding the Causes of the Problem}

After implementing graph algorithms -that are specialized for perfect graphs- to solve real-life problems, the algorithms must be evaluated by testing them with actual perfect graphs. For this purpose, a set of random perfect graphs that can represent the whole set of perfect graphs needs to be obtained. Since there is neither a comprehensive perfect graph database nor a random perfect graph generator implementation open to public, we need to create our own set of random perfect graphs by implementing our own random perfect generator.

\subsection{The Requirements}
% Needs and requirements of the system/customer

In the ideal case, the random perfect graph generator that is going to be implemented shall be able to generate all the possible perfect graphs. This means that all the perfect graphs can be generated with a strictly positive probability. Moreover, for the perfect graphs with the same size (number of vertices), the probability of generation should be uniformly distributed, i.e. the generator should be able to generate all perfect graphs of the same size with equal probabilities.

Furthermore, the time required to generate random perfect graphs using the generator should be reasonable.

\subsection{Limitations and Constraints}
% Limitations and constraints (including important environmental/social/legal/ethical/geopolitical considerations)

Several factors can limit the random perfect graph generation process. These factors include:

\begin{description}
\item [\textbf{Computational complexity:}] \hfill \\
The generation of random perfect graphs can be computationally demanding, particularly for large graphs. This limitation arises from the inherent complexity of verifying perfectness, which is a known NP-complete problem, and maintaining other structural properties during the generation process.

\item [\textbf{Bias and randomness:}] \hfill \\
Ensuring an appropriate balance between randomness and bias in the generated graphs can be difficult. Biased generation algorithms may produce graphs that do not represent the true distribution of random perfect graphs, limiting the applicability and generalizability of the generated graphs.

\item [\textbf{Memory:}] \hfill \\
Generating and manipulating large graphs may require significant amounts of memory, potentially leading to memory limitations or performance degradation. It is important to consider memory management techniques, data compression methods, or other optimizations to ensure efficient use of available memory resources. 
\end{description}

Ideally, the generation process should aim to minimize computation time, memory usage and bias, ensuring that time required for generation remains within practical limits and the generated graphs are a fair and representative sample from the space of all possible random perfect graphs.

Other than these, it is worth noting that since the problem is purely mathematical, there is no environmental, social, legal, ethical, or geopolitical constraints affecting this problem.

\newpage

\subsection{Data}
% Data gathered and used in the identification phase

The generator can be designed to produce a perfect graph from scratch. This approach does not require any data, but one of the possible approaches is generating a graph using other graph(s). Because of this, Brendan McKay's combinatorial data is obtained online~\cite{mckay}, which includes a database of different classes of graphs, including eulerian, chordal, perfect graphs etc.

\subsection{Context Diagram}
% A context diagram (a systemic view) of the handled design problem

\begin{figure}[h]
    \centering
    \adjustbox{max width=\textwidth}{%
    \begin{tikzpicture}
    [every node/.style={align=center},
    Arrow/.style={->, thick},
    TextNode/.style={fill=red!10, rounded corners=5pt},
    Limit/.style={->, thick, densely dotted}]

        %% Nodes %%
        \node [draw, ellipse] (rpgg) {Random Perfect\\Graph Generator};
        \node [right = 15mm of rpgg] (rpg) {Random\\Perfect Graph};
        \node [above left = 1cm and 15mm of rpgg.west, anchor=south] (in1) {Density\\Size};
        \node [below left = 1cm and 15mm of rpgg.west, anchor=north] (in2) {Graphs};
        \node [left = 2cm of rpgg, TextNode] (input) {Input};
        \node [above = 2mm of rpg, TextNode] (output) {Output};

        \coordinate[above = 15mm of rpgg] (up);
        \node[left = 2cm of up, anchor=south] (time) {Time};
        \node[right = 2cm of up, anchor=south] (random) {Randomness};
        \node[above = 5mm of up, anchor=south] (memory) {Memory};
        \node[above = 2mm of memory, TextNode] (limitation) {Limitations};
        %% Arrows %%
        % Input:
        \draw[Arrow] (in1) to[out=270, in=125] (rpgg.165);
        \draw[Arrow] (in2) to[out=90, in=-125] (rpgg.195);
        % Output:
        \draw[Arrow] (rpgg.east) -- (rpg);
        % Limitations:
        \draw[Limit] (time) to[out=270,in=90] (rpgg.120);
        \draw[Limit] (random) to[out=270,in=90] (rpgg.60);
        \draw[Limit] (memory) -- (rpgg);


        \node[below = 2mm of in2] {
        \begin{tikzpicture}[
        anchor=north,
        red/.style={circle,fill=Red!30},
        green/.style={circle,fill=Green!30},
        blue/.style={circle,fill=Blue!30},
        brown/.style={circle,fill=Brown!30},
        salmon/.style={circle,fill=Salmon!30},
        orange/.style={circle,fill=Orange!30},
        cyan/.style={circle,fill=Cyan!30},
        magenta/.style={circle,fill=Magenta!30}]
            \graph[spring layout,
                nodes={draw, circle, as=, inner sep=2.5pt},
                node distance=6mm
            ]
            {
                a[red] --
                {b[green],
                c[blue] -- {h[brown], i[salmon]}, d[orange] -- {e[cyan], f[magenta]}}
            };
        \end{tikzpicture}};
        
        \node[below = 2mm of rpg] {
        \begin{tikzpicture}[
        anchor=north,
        red/.style={circle,fill=Red!30},
        green/.style={circle,fill=Green!30},
        cyan/.style={circle,fill=Cyan!30},
        brown/.style={circle,fill=Brown!30},
        magenta/.style={circle,fill=Magenta!30},
        orange/.style={circle,fill=Orange!30}]
            \graph[spring layout,
                nodes={draw, circle, as=, inner sep=2.5pt},
                node distance=6mm
            ]
            {
                a[red] -- {
                b[green],
                c[cyan],
                d[brown] -- {e[magenta],f[orange]}
                } -- f
            };
        \end{tikzpicture}};
        
    \end{tikzpicture}
    }%
    \caption{Context Diagram of the Project}
    \label{fig:contextDiagram}
\end{figure}

\subsection{Performance Criteria and Potential Improvements}
% Performance criteria and potential improvements

The performance criteria for generating random perfect graphs can include factors such as efficiency, accuracy, scalability, and preservation of key graph properties. The random perfect graph generator should
\begin{itemize}
    \item maintain reasonable computation times,
    \item be able to generate all perfect graphs,
    \item handle large graph sizes,
    \item unbiased.
\end{itemize}

\newpage

\section{Analysis for Solution \& Design Methodology}

\subsection{Literature Review}

\subsubsection{Perfect Graphs}

Perfect graphs are a class of graphs that possess a remarkable property called perfectness. A graph is considered perfect if the chromatic number of every induced subgraph is equal to the size of the largest clique in that subgraph. In simpler terms, a perfect graph has the property that the minimum number of colors required to color its vertices (chromatic number) is equal to the size of the largest complete subgraph (clique) within it.

Perfect graphs have several important properties and characterizations. They are free of certain common subgraphs, such as odd cycles of length at least 5 and their complements (known as Berge graphs). Additionally, perfect graphs can be characterized by Chudnovsky et al.'s strong perfect graph theorem~\cite{spgt}, which states that a graph is perfect if and only if neither the graph nor its complement contains an odd cycle of length at least 5 as an induced subgraph. Therefore, determining whether a graph is perfect can be done by identifying the presence of an odd hole, which can be found either in the graph itself or in its complement. In other words, recognizing the perfectness of a graph is equivalent to identifying whether there exists an odd hole in the graph or its complement.

In 1963, Claude Berge coined the term perfect graph, and he conjectured both the perfect graph theorem and the strong perfect graph theorem~\cite{berge}. 

\subsubsection{Random Perfect Graph Generation}

Creating random graphs in general is a starting point in the domain. The Erdős-Rényi model is a method for creating random graphs given the density (ratio of the number of edges in a graph to the total number of possible edges in that graph) and number of vertices of a graph~\cite{erdos}. This model selects an edge at random and adds it to the graph with a certain probability. The goal of generating a random graph is to ensure that every graph has a non-zero probability of being generated, and that the probability distribution ideally is uniformly distributed for all possible graphs.
 
Some recognition algorithms for perfect graph class are proposed in the literature. Chudnovsky et. al. proposed a recognition time algorithm for detecting perfect graphs that has a polynomial time complexity~\cite{berge-graph}. However, this algorithm fails to give the difference between an odd hole and an odd anti-hole. This results in the fact that with this algorithm, whether a graph is perfect or not can be decided but the reason behind this result cannot be understood.

In a paper by Chudnovsky, Seymour, and Vuskovic~\cite{odd-hole}, a polynomial-time algorithm for detecting odd holes in graphs is described. This result is significant since in this paper, the reason behind the perfectness or not-perfectness can be explained. This algorithm can differentiate between an odd hole and the complement of the odd hole.

In another paper by Oylum Şeker, Tınaz Ekim, and Z. Caner Taşkın~\cite{tinaz}, a perfect graph generation algorithm called \PerfectGen\ is proposed using some available perfect graphs, which is enumerated up to 11 vertices by Brendan McKay~\cite{mckay} and perfection-preserving operations. The authors created perfect graphs with many vertices using operations that preserve perfection. Their approach involved combining small, already known perfect graphs using one of six specific operations that maintain perfection, resulting in a larger perfect graph. They repeated this process until the desired number of vertices was reached. However, this approach does not guarantee that every perfect graph can be generated uniformly at random.

\subsection{Alternative Approaches}

The perfect graph theory field attracts many researchers and is a highly studied branch of graph theory. Specifically, mathematical theoretical results and structures, and recognition, generation, and coloring algorithms are developed by utilizing the intersection of various graph classes to generate random perfect graphs.

In this project, we consider different alternatives for generating random perfect graphs and evaluating algorithm performance. The two main alternatives explored are the Heuristic Repair Method and the Combinatorial Approach.

The Heuristic Repair Method involves using iterative steps to generate random perfect graphs. It iteratively \textit{repairs} imperfect graphs until a perfect graph is obtained. While this method has the potential to generate every perfect graph with a positive probability, it takes more time due to the iterative nature of the algorithm.

On the other hand,  the Combinatorial Approach uses mathematical properties and known structures of perfect graphs to construct new instances. It is generally faster than the heuristic repair method but cannot guarantee a positive probability of generating every perfect graph, limited by a number of mathematical operations.

By studying these different options, our aim is to evaluate the effectiveness of each approach. This analysis will help researchers to choose the most appropriate method for testing their algorithms and evaluating their performance in the world of perfect graph theory.


\subsection{Assumptions}

We make some assumptions for this study of random perfect graph generation to be made: 

\begin{enumerate}

\item It is assumed that the algorithms devised for the generation of random perfect graphs have been implemented accurately and are operating as intended.
\item The properties and characteristics of perfect graphs are well-defined and comprehensively understood within the context of the research.
\item The comparison between the Heuristic Repair Method and the Combinatorial Approach is based on relevant metrics that effectively capture their respective performance.
\item The heuristic repair approach is assumed to have a positive probability of generating every perfect graph, without specifying the exact probability value and giving proper formal proof.
\item The combinatorial algorithm is assumed to not guarantee a positive probability of generating every perfect graph, although no formal proof or counterexample has been provided.
\item This study's findings are assumed to be applicable for guiding researchers in selecting suitable methodologies to advance the field of perfect graph theory based on their specific requirements.

\end{enumerate}

\newpage

\subsection{Brief Overview of the Selected Approaches}
We decided to implement two different approach that generate random perfect graphs. 

First one is the combinatorial  algorithm, which generates perfect graphs by combining smaller perfect graphs using some special operators that preserve perfectness. Main advantage of this algorithm is it does not require a check for the perfectness condition which is a NP-hard problem.

Second algorithm is the heuristic repair algorithm, which generates perfect graphs by trying to repairing non-perfect graphs to perfect ones. This repair is done by some edge modifications that tries to eliminate possible odd cycles. This approach can generate every realization of perfect graphs with a positive probability but it requires a check for the perfectness condition which is a NP-hard problem.

\subsection{IE Skills/Tools to be Integrated}

In this project, we worked in the field of graph theory. Therefore, a strong understanding of graph theory is crucial for working with perfect graphs and their sub-classes. Knowledge of graph properties, algorithms, and graph representation methods would be essential. By using definitions, proofs and structural results from the graph theory world, we defined our project aim and implemented our algorithms in C++ programming language. So, programming knowledge is also necessary to implement and test the algorithms. Furthermore, we needed to use data analysis and visualization tools to analyze and interpret the generated graph data. This enables the evaluation and comparison of different algorithms. Therefore, this study is a senior project that involves designing and comparing algorithms for graph generation, which is aligned with algorithm development and analysis. 

\newpage

\section{Development of Alternative Solutions}
To generate random perfect graphs, we implemented two alternative methods. In the upcoming sections, these two different alternative solutions are discussed.

\subsection{Combinatorial Algorithm}

In this approach, the main idea is to create a random perfect graph by combining smaller perfect graphs using special operators that preserve perfectness. Drawing inspiration from Tınaz Ekim's paper~\cite{tinaz}, this algorithm starts with an initial pool of perfect graphs sourced from McKay's open source database of basic perfect graphs up to 9 vertices.~\cite{mckay} As illustrated in Figure~\ref{fig:combinatorialAlg}, random graphs are selected from this pool, and a random combination operation from the list of available operations is applied. By utilizing specialized perfectness-preserving operations influenced by Ekim's research, the algorithm always constructs larger graphs that are 100\% perfect. Notably, the resulting perfect graphs are then added back to the pool, enriching the collection for subsequent iterations. This iterative process ensures that the algorithm harnesses the combined knowledge and properties of the perfect graphs in the pool, increasing the diversity and quality of the generated random perfect graphs. In Algorithm~\ref{alg:perfectGen} in Appendix~\ref{sec:pseudo}, the combinatorial algorithm is explained step by step.

\begin{figure}[h]
    \centering
    \adjustbox{max width=\textwidth}{%
    \begin{tikzpicture}
    [Label/.style={rounded corners=5pt, fill=red!10}]

        %% Nodes %%
        \node (G1_l) {$G_1$};
        \node[right = 25mm of G1_l] (G2_l) {$G_2$};

        \node[right = 7cm of G2_l] (G_l) {$G$};
        
        \node[above = 17mm of G1_l, centered] (G1) {
        \begin{tikzpicture}[
        anchor=south,
        red/.style={circle,fill=Red!30},
        salmon/.style={circle,fill=Salmon!30},
        cyan/.style={circle,fill=Cyan!30}]
            \graph[spring layout,
                nodes={draw, circle, fill=black, as=, inner sep=2.5pt},
                node distance=8mm
            ]
            {
                a -- {b, c, e, f -- g};
            };
        \end{tikzpicture}};

        \node[above = 17mm of G2_l, centered] (G2) {
        \begin{tikzpicture}[
        red/.style={circle,fill=Red!30},
        salmon/.style={circle,fill=Salmon!30},
        cyan/.style={circle,fill=Cyan!30}]
            \graph[spring layout,
                nodes={draw, circle, fill=black, as=, inner sep=2.5pt},
                node distance=8mm,
            ]
            {
                a -- {b, c, d -- {e, f -- {b,c}}};
                b -- e;
                
            };
        \end{tikzpicture}};

        \node[above = 17mm of G_l, centered] (G) {
        \begin{tikzpicture}[
        red/.style={circle,fill=Red!30},
        salmon/.style={circle,fill=Salmon!30},
        cyan/.style={circle,fill=Cyan!30}]
            \graph[spring layout,
                nodes={draw, circle, fill=black, as=, inner sep=2.5pt},
                node distance=9mm,
            ]
            {
            a -- b -- c -- d -- a;
            a -- e;
            b -- f;
            c -- {g, h, i};
            d -- {k};
            };
        \end{tikzpicture}};
        
        \draw[-{Stealth[width=10pt,length=15pt]}, thick, shorten >=5mm, shorten <=5mm, line width=4pt] (G2) -- node[midway,above,align=center] {Combinatorial\\Operation} (G);

        \coordinate (left_mid) at ($ (G1_l)!.5!(G2_l) $);
        \node [Label, above = 4cm of left_mid] {2 Perfect Graphs};
        \node [Label, above = 4cm of G_l.mid] {Larger Perfect Graph};
    \end{tikzpicture}
    }%
    \caption{Illustration of Combinatorial Algorithm}
    \label{fig:combinatorialAlg}
\end{figure}

\subsubsection{Combinatorial Operations}

The combinatorial algorithm uses operations defined in~\cite{tinaz}. In the paper, 6 operations are mentioned:

\begin{description}
    \item[Clique Identification] \hfill \\
    Let $G_1$, $G_2$ be disjoint graphs, and $K_i$ be a nonempty clique in $G_i$ such that $|K_1| = |K_2|$. Define a one-to-one mapping between the vertices of $K_1$ and $K_2$. A graph obtained by merging each mapped $v_1$ in $K_1$ with vertex $v_2$ in $K_2$ is said to arise from $G_1$ and $G_2$ by clique identification. A graph $G$ generated via clique identification is \textit{perfect}.

    In Algorithm \PerfectGen\, one vertex from $G$ and one vertex from $G'$ is chosen, and each one is extended to a maximal clique, $K_1$ and $K_2$. Assuming $|K_1| \leq |K_2|$, random $|K_1|$ vertices from $K_2$ are chosen and are identified with those in $K_1$. The one-to-one mapping is randomly determined.

    \begin{figure}[h]
    \centering
    \adjustbox{max width=\textwidth}{%
    \begin{tikzpicture}

        %% Nodes %%
        \node (G1_l) {$G_1$};
        \node[right = 3 of G1_l] (G2_l) {$G_2$};

        \node[right = 4cm of G2_l] (G_l) {$G$};
        
        \node[above = 2cm of G1_l, centered] (G1) {
        \begin{tikzpicture}[
        anchor=south,
        red/.style={circle,fill=Red!30},
        salmon/.style={circle,fill=Salmon!30},
        cyan/.style={circle,fill=Cyan!30},
        subgraph nodes={label=120:$K_1$, dashed, very thin},
        subgraph text none]
            \graph[spring layout,
                nodes={draw, circle, fill=black, as=, inner sep=2.5pt},
                node distance=8mm
            ]
            {
                a[red] -- {b[cyan] -- {c[salmon], d}, c -- e};
                K1 [draw]  // { a, b, c };
            };
        \end{tikzpicture}};

        \node[above = 2cm of G2_l, centered] (G2) {
        \begin{tikzpicture}[
        red/.style={circle,fill=Red!30},
        salmon/.style={circle,fill=Salmon!30},
        cyan/.style={circle,fill=Cyan!30},
        subgraph nodes={label=80:$K_2$, dashed, very thin},
        subgraph text none]
            \graph[spring layout,
                nodes={draw, circle, fill=black, as=, inner sep=2.5pt},
                node distance=8mm,
            ]
            {
                a[cyan] -- {b[red] -- {c[salmon] -- a, d}};
                a -- e -- c;

                $K_2$ [draw] // { a, b, c };
            };
        \end{tikzpicture}};

        \node[above = 2cm of G_l, centered] (G) {
        \begin{tikzpicture}[
        red/.style={circle,fill=Red!30},
        salmon/.style={circle,fill=Salmon!30},
        cyan/.style={circle,fill=Cyan!30},]
            \graph[spring layout,
                nodes={draw, circle, fill=black, as=, inner sep=2.5pt},
                node distance=12mm,
            ]
            {
                a[red] -- b[cyan] -- c[salmon] -- a;
                e -- {b -- f, c -- g};
                a -- h;
            };
        \end{tikzpicture}};
        
        \draw[->, thick, shorten >=5mm, shorten <=5mm] (G2) -- (G);
    \end{tikzpicture}
    }%
    \caption{Clique Identification}
    \label{fig:cliqueIdentificationExample}
\end{figure}
    
    \item[Substitution] \hfill \\
    Let $G_1$, $G_2$ be disjoint graphs, $v$ be a vertex of $G_1$, and $N$ the set of all neighbors of $v$ in $G_1$. Removing $v$ from $G_1$ and linking each vertex in $G_2$ to those in $N$ results in a graph that arises from $G_1$ and $G_2$ by substitution. If $G_1$ and $G_2$ are perfect, a graph $G$ obtained via substitution of these graphs is also \textit{perfect}. We note that this operation is also known as Replication Lemma in the literature and it played an important role in the proof of the WPGT\footnote{Weak Perfect Graph Theorem}. The combinatorial algorithm randomly picks a vertex $v$ from $G_1$, and then substitutes $v$ with $G_2$ as explained above.

    \begin{figure}[h]
    \centering
    \adjustbox{max width=\textwidth}{%
    \begin{tikzpicture}

        %% Nodes %%
        \node (G1_l) {$G_1$};
        \node[right = 3cm of G1_l] (G2_l) {$G_2$};

        \node[right = 4cm of G2_l] (G_l) {$G$};
        
        \node[above = 15mm of G1_l, centered] (G1) {
        \begin{tikzpicture}[
        anchor=south,
        red/.style={circle,fill=Red!30},
        salmon/.style={circle,fill=Salmon!30},
        cyan/.style={circle,fill=Cyan!30},
        subgraph nodes={label=120:$N$, dashed, very thin},
        subgraph text none]
            \graph[simple necklace layout,
                nodes={draw, circle, fill=black, as=, inner sep=2.5pt},
                node distance=10mm
            ]
            {
                a -- b -- c -- d[red, label=right:$v$] -- a;
                K1 [draw]  // { a, c };
            };
        \end{tikzpicture}};

        \node[above = 15mm of G2_l, centered] (G2) {
        \begin{tikzpicture}[
        red/.style={circle,fill=Red!30},
        salmon/.style={circle,fill=Salmon!30},
        cyan/.style={circle,fill=Cyan!30}]
            \graph[spring layout,
                nodes={draw, circle, fill=black, as=, inner sep=2.5pt},
                node distance=8mm,
                vertical= a to b
            ]
            {
                a -- b -- c -- a;
            };
        \end{tikzpicture}};

        \node[above = 15mm of G_l, centered] (G) {
        \begin{tikzpicture}[
        red/.style={circle,fill=Red!30},
        salmon/.style={circle,fill=Salmon!30},
        cyan/.style={circle,fill=Cyan!30},
        subgraph nodes={label=120:$N$, dashed, very thin},
        subgraph text none]
            \graph[spring layout,
                nodes={draw, circle, fill=black, as=, inner sep=2.5pt},
                node distance=5mm,
                horizontal = a to f
            ]
            {
                a -- {b,c};
                b -- {d[nudge down = 2.2mm],e[nudge up = 2.2mm],f[nudge right = 12mm]};
                c -- {d,e,f};
                d -- e -- f -- d;
                N [draw]  // { b, c };
            };
        \end{tikzpicture}};
        
        \draw[->, thick, shorten >=3mm, shorten <=3mm] (G2) -- (G);
    \end{tikzpicture}
    }%
    \caption{Substitution}
    \label{fig:substitutionExample}
\end{figure}

    \item[Composition] \hfill \\
    Let $G_1$, $G_2$ be disjoint graphs each with at least three vertices, $v_i$ be a vertex of $G_i$, $N(v_i)$ the set of all neighbors of $v_i$. The composition of $G_1$ and $G_2$ is obtained from $G_1 \setminus {v_1}$ and $G_2 \setminus {v_2}$ by linking all vertices in $N(v_1)$ to those in $N(v_2)$. A graph obtained from two perfect graphs via composition is \textit{perfect}.

    \begin{figure}[h]
    \centering
    \adjustbox{max width=\textwidth}{%
    \begin{tikzpicture}

        %% Nodes %%
        \node (G1_l) {$G_1$};
        \node[right = 3cm of G1_l] (G2_l) {$G_2$};

        \node[right = 4cm of G2_l] (G_l) {$G$};
        
        \node[above = 15mm of G1_l, centered] (G1) {
        \begin{tikzpicture}[
        red/.style={circle,fill=Red!30},
        salmon/.style={circle,fill=Salmon!30},
        cyan/.style={circle,fill=Cyan!30},
        subgraph nodes={label=120:$N(v_1)$, dashed, very thin},
        subgraph text none]
            \graph[simple necklace layout,
                nodes={draw, circle, fill=black, as=, inner sep=2.5pt},
                node distance=10mm
            ]
            {
                a -- b -- c -- d[red, label=right:$v_1$] -- a;
                N1 [draw]  // { a, c };
            };
        \end{tikzpicture}};

        \node[above = 15mm of G2_l, centered] (G2) {
        \begin{tikzpicture}[
        red/.style={circle,fill=Red!30},
        salmon/.style={circle,fill=Salmon!30},
        cyan/.style={circle,fill=Cyan!30},
        subgraph nodes={label=120:$N(v_2)$, dashed, very thin},
        subgraph text none]
            \graph[spring layout,
                nodes={draw, circle, fill=black, as=, inner sep=2.5pt},
                node distance=8mm,
                horizontal=b to d
            ]
            {
                a[red, label=left:$v_2$] -- b -- c -- a;
                b -- d -- e -- c;
                N2 [draw]  // { b, c };
            };
        \end{tikzpicture}};

        \node[above = 15mm of G_l, centered] (G) {
        \begin{tikzpicture}[
        red/.style={circle,fill=Red!30},
        salmon/.style={circle,fill=Salmon!30},
        cyan/.style={circle,fill=Cyan!30},
        subgraph nodes={label=120:$N$, dashed, very thin},
        subgraph text none]
            \graph[spring layout,
                nodes={draw, circle, fill=black, as=, inner sep=2.5pt},
                node distance=9mm,
                vertical = g to f,
            ]
            {
            a -- {b,c};
            d -- e -- f -- g -- d;
            b -- {d,e};
            c -- {d,e};
            };
        \end{tikzpicture}};
        
        \draw[->, thick, shorten >=5mm, shorten <=5mm] (G2) -- (G);
    \end{tikzpicture}
    }%
    \caption{Composition}
    \label{fig:compositionExample}
\end{figure}
    
    \item[Disjoint Union] \hfill \\
    Let $G_1$, $G_2$ be two disjoint graphs. The disjoint union of $G_1$ and $G_2$ is simply $G = G_1 \cup G_2$ with $V(G) = V(G_1) \cup V(G_2)$ and $E(G) = E(G_1) \cup E(G_2)$. Disjoint union of perfect graphs is \textit{perfect}.

    \item[Join] \hfill \\
    Let $G_1$, $G_2$ be disjoint graphs. The join of $G_1$ and $G_2$, is obtained by connecting all nodes in $G_1$ to all those in $G_2$. A graph generated from two perfect graphs via join operation is perfect. As a proof, assume that $G_1$ and $G_2$ are perfect. Consider $\bar{G}$ which is simply $\bar{G_1} \cup \bar{G_2}$. $G_1$ and $G_2$ are perfect, so $\bar{G_1}$ and $\bar{G_2}$ are perfect, too (WPGT). Like disjoint union, $\bar{G} = \bar{G_1} \cup \bar{G_2}$ is \textit{perfect}.

    \item[Complement] \hfill \\
    By WPGT, the complement of a perfect graph is again \textit{perfect}.

\end{description}

\subsubsection{Strengths and Weaknesses of the Combinatorial Algorithm}
\paragraph{Strengths:}

\begin{itemize}
\item \textbf{Efficient Generation:} Combinatorial algorithm allows for the quick creation of random perfect graphs by utilizing perfectness-preserving operations. By bypassing the need for perfectness verification, the algorithm avoids the computationally demanding task, which is known to be NP-complete. This results in significant time savings during the graph generation process.
\end{itemize}
\paragraph{Weaknesses:}

\begin{itemize}
\item \textbf{Limited Scope:} Combinatorial algorithm's pool generation method, which relies on iteratively calling the algorithm and growing the pool, introduces a factor that can impact the probabilities of generating different graphs. The order in which the individual runs occur affects how the pool grows, potentially leading to biases in the generated random graphs. Consequently, the algorithm's pool generation method may limit the diversity and range of perfect graphs that can be generated.

\item \textbf{Incompleteness:} Due to the limited knowledge about all possible perfect graphs, it remains unknown whether \PerfectGen\ can generate all potential perfect graphs. The algorithm's dependence on combining and iterating from the initial pool might result in some perfect graph configurations being unattainable through this approach.
\end{itemize}

In summary, combinatorial algorithm offers efficient generation of random perfect graphs by utilizing perfectness-preserving operations. It avoids the time-consuming task of perfectness verification while utilizing existing knowledge from a pool of initial perfect graphs. This way, this method is able to generate very large perfect graphs quickly; however, its limitations include a restricted scope based on the initial pool and the uncertainty of generating all possible perfect graphs.

\subsection{Heuristic Repair Algorithm}
\subsubsection{Methodology Design}
When designing algorithms, it is often necessary to have a mechanism for exploring the vertices and edges of a graph. By having access to the adjacency sets, we can repeatedly move from a vertex to one of its neighboring vertices, allowing us to navigate through the graph. During this exploration process, some vertices will have already been visited, while others have not. The next vertex to visit, denoted as \textit{x}, must be chosen, and since there may be multiple eligible candidates for \textit{x}, it becomes important to establish a priority among them.

This section discusses two priority criteria that are particularly useful for graph exploration: \textbf{depth-first search (DFS)} and \textbf{breadth-first search (BFS)}. In both methods, each edge is traversed once in both the forward and reverse directions, and every vertex is visited. By examining a graph in such a structured manner, certain algorithms become easier to comprehend and execute more efficiently. The choice between DFS and BFS can significantly impact the algorithm's efficiency, meaning that selecting an appropriate data structure alone is not enough to ensure a well-performing implementation. A carefully chosen search technique is also essential. 

For this reason, in our heuristic repair algorithm, we used BFS technique to explore the starting graph and decide the repair moves in each iteration. The idea is to pick up a starting graph and make some clever and random edge changes to disturb possible odd cycles in this method. Therefore, in this section, we explain the disturbing phase while in the second part, we discuss the starting graph selection. Firstly, we divide this first design into two parts: \textbf{Starting Vertex Approach} and \textbf{Parent Vertex Approach}.

These two approaches, \texttt{Parent Vertex Approach in Heuristic Re\-pair Al\-go\-rithm} and \texttt{Starting Vertex Approach in Heuristic Repair Al\-go\-rithm}, are explained in Algorithm~\ref{alg:parentvertex} and Algorithm~\ref{alg:startingvertex} in Appendix~\ref{sec:pseudo}, respectively.

\input{images/heuristicRepairAlgorithm/startingVertexApproach}

\begin{figure}[H]
    \centering
    \adjustbox{max width=\textwidth}{%
    \begin{tikzpicture}
    [Description/.style={align=center, text width=4cm, anchor=north west, rounded corners=5pt, fill=gray!10, draw, inner sep=3mm}]
        %% Nodes %%
        \node[Description] (G1_l) {Generate the \textit{BFS tree} of the current with a randomly selected starting vertex};
        \node[right = 1cm of G1_l.north east, Description] (G2_l) {For each vertex $v$, check if $v$ has an edge between vertices that shares the same level in BFS tree 
};
        \node[right = 1cm of G2_l.north east, Description] (G3_l) {Remove the edge between $v$ and its parent. Then add an edge between $v$’s neighbor and $v$’s parent.
};
        
        \node[above = 2cm of G1_l, centered] (G1) {
        \tikz [tree layout, minimum number of children=2,
       sibling distance=8mm, level distance=7mm]
  \graph [nodes={circle, inner sep=0pt, minimum size=2mm, fill, as=}, edges={thick}]{
    a[fill=Red!80!black] -- { b -- c -- { d -- e, f -- { g, h }}, i -- j -- k[second] }
  };};

  \node[above = 2cm of G2_l, centered] (G2) {
        \tikz [tree layout, minimum number of children=2,
       sibling distance=8mm, level distance=7mm]
  \graph [nodes={circle, inner sep=0pt, minimum size=2mm, fill, as=}, edges={thick}]{
    a[fill=Red!80!black] -- { b -- c[fill=RoyalBlue] -- { d -- e, f -- { g, h }}, i -- j -- k[second] };
    c --[color=Red!80!black] j;
  };};

    \node[above = 2cm of G3_l, centered] (G3) {
        \begin{tikzpicture}
            \coordinate (a);
            \coordinate (j);
            \graph [tree layout, minimum number of children=2, sibling distance=8mm, level distance=7mm, nodes={circle, inner sep=0pt, minimum size=2mm, fill, as=}, edges={thick}]{
                a[fill=Red!80!black] -- { b --[dashed] c[fill=RoyalBlue] -- { d -- e, f -- { g, h }}, i -- j -- k[second] };
                c --[color=Red] j;
            };
            \draw[thick, color=Green] (b) -- (j);
        \end{tikzpicture}
    };
    
    \draw[-{Stealth[width=10pt,length=15pt]}, thick, shorten >=7mm, shorten <=7mm, line width=4pt] (G1) -- (G2);
    \draw[-{Stealth[width=10pt,length=15pt]}, thick, shorten >=7mm, shorten <=7mm, line width=4pt] (G2) -- (G3);
  
    \end{tikzpicture}
    }%
    \caption{Parent Vertex Approach}
    \label{fig:parentVertexApproach}
\end{figure}

\subsubsection{Starting Graph Design}

In the above-mentioned algorithm, a graph is taken as input. This graph can be a random graph with a specific size and density. The first method from the literature to generate random graphs is Erdős - Rényi random graph model ~\cite{erdos}. However, from the literature, we know that almost no ER random graph will be perfect ~\cite{erdos}. Still, we generated some ER random graph and use these graphs as input starting graphs. At the end, we conducted some tests regarding the performance of this method. Even though it will be discussed, the result can be explained shortly that this method does not generate perfect graphs as many as the other approach does. Therefore, we need to select a better starting graph to generate random perfect graphs. From this point, we selected a certain graph class, ${C_5-free}$ graphs, where the smallest component that violates the perfectness is ${C_5}$ (cycle of length 5). The reason of this selection comes from a result from the literature: Almost all ${C_5-free}$ graphs are perfect when the number of vertices goes to infinity ~\cite{promel}. Therefore, this result might result in an improvement in our random perfect graph generation process. We decided to run our edge-modification heuristic repair algorithm by using ${C_5-free}$ graphs as starting graphs to increase the speed of the algorithm by decreasing the number of iterations to generate a perfect graph. This idea is reasonable since the bottleneck of the algorithm is perfectness check and the only way to decrease the time spent in perfectness recognition is to decrease the number of iterations, where in each iteration, perfectness check is conducted. Finding a faster recognition algorithm is beyond the scope of this study. Therefore, we implemented ${C_5-free}$ graph generation and recognition algorithm to generate random ${C_5-free}$ graphs to be used in the heuristic repair algorithm. As in Algorithm~\ref{alg:c5check}, whether a graph is ${C_5-free}$ or not can be recognized; while with the Algorithm~\ref{alg:c5gen}, we can generate ${C_5-free}$ graphs in general. These algorithms can be found in Appendix~\ref{sec:pseudo}.

%Here, we need to put C5-free generation and checker algo

Furthermore, we searched for a way to start with a better starting graphs to generate random perfect graphs. We get a result from the literature, where a new type of graph class section \textit{clean odd cycle-free graphs} helps us get perfect graphs faster. Also, there had been some studies for clean odd-cycle graph generation and checker implementation in Bogazici University. Therefore, it is expected to get some better results; however, there is still some time needed to implement this third phase of the project and this improvement idea could be a next step for the upcoming studies. Still, the idea of recognizing \textit{clean odd cycle-free graphs} is explained with the Algorithm~\ref{alg:cleanrec} and generation algorithm is presented in the Algorithm~\ref{alg:cleangen} in Appendix~\ref{sec:pseudo}.

\subsubsection{Strengths and Weaknesses of the Heuristic Repair Method}
\paragraph{Strengths:}

\begin{itemize}
\item \textbf{Potential for Generating All Perfect Graphs:} The Heuristic Repair Method has the potential to generate every perfect graph with a positive probability. By iteratively repairing and modifying initially imperfect graphs, this approach explores different configurations, increasing the chances of generating a wide variety of perfect graphs.
\item \textbf{Flexibility and Adaptability:} The heuristic nature of the algorithm allows for flexibility in exploring different repair strategies and adapting to the characteristics of each specific graph. This adaptability enhances the algorithm's ability to generate diverse perfect graph instances.
\end{itemize}

\paragraph{Weaknesses:}

\begin{itemize}
\item \textbf{Time-Intensive Process:} The Heuristic Repair Method is computationally more demanding and time-consuming compared to the Combinatorial Approach. The iterative nature of the algorithm, involving repairs and modifications, and perfectness check, increases the time required to generate a perfect graph. As a result, the overall efficiency of the algorithm may be compromised.
\item \textbf{Lack of Formal Proof:} Although the Heuristic Repair Method has the potential to generate every perfect graph, there is no formal proof or counterexample provided to verify this claim. The algorithm's performance in generating specific perfect graph configurations remains dependent on its heuristics and repair strategies.
\end{itemize}

In summary, the Heuristic Repair Method can generate various perfect graphs by iteratively repairing and modifying imperfect graphs. It is flexible and adaptable, leading to the generation of diverse perfect graph instances. However, it is more time-consuming than the Combinatorial Approach, and its ability to generate all possible perfect graphs is uncertain due to the lack of formal proof.

\newpage

\section{Comparison of Alternatives and Recommendations}
In this section, we discuss different alternatives and their performances in terms of speed, memory and limitations. In order to test the algorithms, the algorithms are implemented in C++, with a graph library called ``\texttt{igraph}''~\cite{igraph}. For those who are interested, the implementations can be found in our GitHub repository~\cite{repository}.

\subsection{Heuristic Algorithm}

\begin{table}[H]
\centering
\caption{Comparison of C$_5$-free and Erdős–Rényi Graphs in terms of Perfect Graph Generation Time in Seconds}
\label{tab:ER-compared}
\adjustbox{max width=\textwidth}{%
\begin{tabular}{|c|cc|cc|}
\hline
        & \multicolumn{2}{c|}{\textbf{Erdös-Rényi}}                    & \multicolumn{2}{c|}{\textbf{C$_5$-free}}                                 \\ \hline
\textbf{Density} & \multicolumn{1}{c|}{\textbf{Size 60}}             & \textbf{Size 100} & \multicolumn{1}{c|}{\textbf{Size 60}}             & \textbf{Size 100}          \\ \hline
0.1     & \multicolumn{1}{c|}{10,504 (214 graphs)} & --       & \multicolumn{1}{c|}{17,463 (205 graphs)} & --                \\ \hline
0.3     & \multicolumn{1}{c|}{--}                  & --       & \multicolumn{1}{c|}{10,812 (225 graphs)} & 1,468 (45 graphs) \\ \hline
0.5     & \multicolumn{1}{c|}{--}                  & --       & \multicolumn{1}{c|}{10,650 (2 graphs)}   & 4,445 (1 graph)   \\ \hline
\end{tabular}
}
\end{table}

For the heuristic algorithm, average perfect graph generation time of Erdős–Rényi model and C$_5$-free method are compared in Table~\ref{tab:ER-compared}. Generating initial graph with Erdős–Rényi Model does not generate perfect graphs after 60 vertices with 0.1 density. Generating initial graph with C$_5$-free method is able to generate bigger perfect graphs. The reason is the C$_5$ free graphs are closer to perfect graphs, so the repair algorithm is able to reach a perfect graph, even for larger sizes.

\begin{table}[H]
\centering
\caption{Average Computation Time of Generating \& Recognizing Graphs in Seconds (Using ${C_5-free}$ Method)}
\label{tab:bottleneck}
\adjustbox{max width=\textwidth}{%
\begin{tabular}{|c|cc|cc|}
\hline
        & \multicolumn{2}{c|}{\textbf{Size 60}}                  & \multicolumn{2}{c|}{\textbf{Size 100}}                 \\ \hline
\textbf{Density} & \multicolumn{1}{c|}{\textbf{Generation}} & \textbf{Recognition} & \multicolumn{1}{c|}{\textbf{Generation}} & \textbf{Recognition} \\ \hline
0.1     & \multicolumn{1}{c|}{72}         & 87          & \multicolumn{1}{c|}{--}         & --          \\ \hline
0.3     & \multicolumn{1}{c|}{54}         & 81          & \multicolumn{1}{c|}{55}         & 801         \\ \hline
\end{tabular}
}
\end{table}

Next, another important comparison is the time required to generate graphs and verify their perfectness, which is shown in Table~\ref{tab:bottleneck}. The generation time of the initial random graph scales much less compared to the total time for checking the perfectness condition. As can be seen in the table, the recognition algorithm is the \textit{bottleneck} for this method, preventing the algorithm from creating larger graphs.

\begin{table}[H]
\centering
\caption{Mean Computation Time of Perfectness Recognition Algorithms with Various Graph Sizes and Densities}
\label{tab:recognition}
\begin{tabular}{|c|cc|cc|}
\hline
        & \multicolumn{2}{c|}{\textbf{Size 50}}           & \multicolumn{2}{c|}{\textbf{Size 100}}          \\ \hline
\textbf{Density} & \multicolumn{1}{c|}{\textbf{Oylum}} & \textbf{\texttt{IsPerfect}} & \multicolumn{1}{c|}{\textbf{Oylum}} & \textbf{\texttt{IsPerfect}} \\ \hline
0.2     & \multicolumn{1}{c|}{51}    & 23        & \multicolumn{1}{c|}{--}    & --        \\ \hline
0.4     & \multicolumn{1}{c|}{--}    & --        & \multicolumn{1}{c|}{801}   & 55        \\ \hline
\end{tabular}
\end{table}

Furthermore, after deciding that graph recognition is the bottleneck of the heuristic approach, different perfect graph recognition algorithms are tested to find the algorithm that performs the fastest on the average. When implementing the algorithms, we have encountered 2 different recognition algorithms:
\begin{itemize}
    \item \texttt{isPerfect} from \texttt{igraph} library
    \item Oylum Şeker's algorithm focuses on finding odd holes, works fast on imperfect graphs but slow on perfect graphs.
\end{itemize}
Table~\ref{tab:recognition} compares these 2 recognition algorithms. According to the table, \texttt{igraph}'s \texttt{IsPerfect} method performs faster when given a perfect graph. So, we run and test our implementations with \texttt{IsPerfect} method.

\subsection{Combinatorial Algorithm}

The combinatorial algorithm is run thousands of times to create a large pool of perfect graphs. First, the graph sizes $n_1$ and $n_2$ are chosen 5. With the chosen input, 200 graphs are created. Then $n_1$ and $n_2$ are increased with the following rule:

\begin{quote}
If $n_1$ and $n_2$ are equal, increment $n_1$ by 1 and set $n_2$ as 5. Otherwise, increment $n_2$ by 1.
\end{quote}

With this rule, the algorithm is run in a loop with the following sequence of input pairs $n_1$-$n_2$:5-5, 6-5, 6-6, 7-5, 7-6, 7-7\hspace{1em}$\cdots$\hspace{1em} until 55-55; and with each pair, 200 random perfect graphs are generated.

In total, 241,486 perfect graphs of various sizes are generated. The distribution of these graphs are as follows:

\begin{itemize}
    \item 134,027 disconnected graphs
    \item 107,459 connected graphs
    \item 17 acyclic graph
    \item 44 bipartite graphs
    \item 907 chordal graphs
\end{itemize}

It is worth noting that the acyclic and bipartite graphs have small sizes (< 25), they are not encountered when the size increases. This is because these graph classes are very special and the random selection of the combinatorial operation decreases the probability of generating them when the size increases. Chordal graphs are generated more frequently with this algorithm, but their sizes are also small. Almost half of the graphs are connected, which tells us that with this approach, it is equally likely to generate a connected or disconnected graph. Calculating the bias of this algorithm requires extensive analysis, but we are content with the fact that there is no extreme accumulation towards a specific graph class.

\begin{figure}[H]
    \centering
    \includesvg[width=\textwidth]{plots/timePerSize}
    \caption{Combinatorial Algortihm's Mean Graph Generation Time per Graph Size}
    \label{fig:comb-alg-timePerSize}
\end{figure}

As it can be seen in Figure~\ref{fig:comb-alg-timePerSize}, there is a nonlinear behaviour in large degrees as the graph size increases, mean generation time increases in such shape. Even if it is not linear, still, the time required to generate a perfect graph using combinatorial algorithm is still much smaller than the heuristic repair generation algorithm. On average, it does not even take 0.04 seconds to create a perfect graph of size 100 with this algorithm. On the contrary, the heuristic requires almost a minute to create a perfect graph with 100 vertives. This is an expected result of the combinatorial algorithm; as mentioned in the previous section, the main advantage of the combinatorial algorithm is that it skips the time consuming perfectness check.

Next, the distribution of clique numbers considering different parameters such as graph size and graph density are investigated.  Firstly, we studied the graphs of \textit{size of 50} with different \textit{graph density} values such as ${0.2}$, ${0.5}$ and ${0.8}$. The distribution of these graphs looks reasonable when taking the shape of the distribution plot into account. 

\begin{figure}[H]
    \centering
    \includesvg[width=\textwidth]{plots/cliqueNumber_50_2}
    \caption{Distribution of Clique Numbers of Graphs of Size 50 and Density 0.2}
    \label{fig:comb-alg-clique-50-2}
\end{figure}

\begin{figure}[H]
    \centering
    \includesvg[width=\textwidth]{plots/cliqueNumber_50_5}
    \caption{Distribution of Clique Numbers of Graphs of Size 50 and Density 0.5}
    \label{fig:comb-alg-clique-50-5}
\end{figure}

\begin{figure}[H]
    \centering
    \includesvg[width=\textwidth]{plots/cliqueNumber_50_8}
    \caption{Distribution of Clique Numbers of Graphs of Size 50 and Density 0.8}
    \label{fig:comb-alg-clique-50-8}
\end{figure}

Then, this form continues for the graphs of size 70. However, as the size of the graph increases, the clique number distribution shifts towards larger values. 

\begin{figure}[H]
    \centering
    \includesvg[width=\textwidth]{plots/cliqueNumber_70_2}
    \caption{Distribution of Clique Numbers of Graphs of Size 70 and Density 0.2}
    \label{fig:comb-alg-clique-70-2}
\end{figure}

\begin{figure}[H]
    \centering
    \includesvg[width=\textwidth]{plots/cliqueNumber_70_5}
    \caption{Distribution of Clique Numbers of Graphs of Size 50 and Density 0.5}
    \label{fig:comb-alg-clique-70-5}
\end{figure}

\begin{figure}[H]
    \centering
    \includesvg[width=\textwidth]{plots/cliqueNumber_70_8}
    \caption{Distribution of Clique Numbers of Graphs of Size 50 and Density 0.8}
    \label{fig:comb-alg-clique-70-8}
\end{figure}

In the end, when we considered random perfect graphs of size 100, the behaviour changes and the distribution of the clique numbers of these perfect graphs starts to differ in a wider range. Therefore, as the number of vertices in the graph increases, the random behaviour starts to take place as expected. In the upcoming figures ~\ref{fig:comb-alg-clique-100-2}, ~\ref{fig:comb-alg-clique-100-5} and ~\ref{fig:comb-alg-clique-100-8}, this randomness can be observed.

\begin{figure}[H]
    \centering
    \includesvg[width=\textwidth]{plots/cliqueNumber_100_2}
    \caption{Clique Numbers of Graphs of Size 100, Density 0.2}
    \label{fig:comb-alg-clique-100-2}
\end{figure}

\begin{figure}[H]
    \centering
    \includesvg[width=\textwidth]{plots/cliqueNumber_100_5}
    \caption{of Clique Numbers of Graphs of Size 100, Density 0.5}
    \label{fig:comb-alg-clique-100-5}
\end{figure}

\begin{figure}[H]
    \centering
    \includesvg[width=\textwidth]{plots/cliqueNumber_100_8}
    \caption{Clique Numbers of Graphs of Size 100, Density 0.8}
    \label{fig:comb-alg-clique-100-8}
\end{figure}

In summary, considering the bar plots of clique numbers, which is the size of the largest clique that can be made up of edges and vertices of ${G}$, we gathered data for different graphs having different size and density values. We find out that the combinatorial algorithm can generate in different values of size, density and clique number. This shows that even if we expected a limitation of this algorithm by a limited number of operations intuitively without a formal proof, we get decent perfect graphs using the algorithm. Also, with this behaviour, this algorithm does not guarantee to generate every perfect graph with a positive probability, still, practically, it seems to be useful to test and evaluate the algorithms. When considering that the combinatorial algorithm is faster than the heuristic random graph repair generation algorithm, this approach can be applicable for many problems.  

\newpage

\section{Suggestions for a Successful Implementation}

For a successful implementation of the algorithms, we can list some suggestions to improve the effectiveness and efficiency of the algorithms:

\begin{itemize}

\item \textbf{Algorithmic Complexity Analysis:} A thorough analysis of the algorithm's time complexity should be conducted, and potential bottlenecks should be identified. Focus should be placed on optimizing the most time-consuming operations, and alternative algorithms or data structures that may provide better efficiency for specific steps should be explored.

\item \textbf{Parallelization:} Opportunities for parallelization should be explored to leverage multiple processors or threads for faster execution. Parallel processing techniques, such as multi-threading or distributed computing, can be implemented to accelerate the generation of random perfect graphs.

\item \textbf{Pool Diversity:} The pool of perfect graphs used in the combinatorial algorithm should be diverse and representative of a wide range of graph structures and properties. A more diverse initial pool increases the chances of generating a broader range of random perfect graphs.

\item \textbf{Parameter Tuning} Incorporating adjustable parameters in the combinatorial algorithm can be considered to allow fine-tuning of the generated random perfect graphs. For example, parameters controlling the selection and combination of graphs from the pool can be adjusted to influence the diversity and randomness of the generated graphs.
\end{itemize}

\newpage

\section{Conclusions and Discussion}

In this project, C++ is used to implement and test the algorithms which try to generate perfect graphs as random as possible. This requires the application of fundamental concepts from industrial engineering at the undergraduate level, including graph theory and the algorithm design. 

Graph Theory is a mathematical field focusing on the mathematical structures, namely graphs which have many real-life applications. Furthermore, perfect graph class is a special structured graph class, where the perfect class and its sub-classes are well-studied. There are different algorithms developed for such graph classes, which might be in polynomial time, approximation or heuristic algorithms. It is necessary to have a list of random graphs to test and evaluate the performance of these algorithms. Therefore, it is convenient to generate random perfect graphs in different size and density. For our study, we implemented two different algorithms. Firstly, we proposed a combinatorial algorithm using a open-source pool source of perfect graphs and known perfectness-preserving operations. Then, we proposed a heuristic repair algorithm having either a random graph, or a $C_{5}$-free graph as starting graph and making some edge modifications (repair) to finally result in a perfect graph. At the end, we made comparisons between the algorithms. 

Finally, we concluded that we can generate as many perfect graphs as possible in different size and density by using the combinatorial algorithm since this approach is fast. However, the drawback of this method is that this method is limited by a number operations and cannot generate every perfect graph with positive probability. On the other hand, our heuristic repair approach is better since we can generate perfect graphs with positive probability; however, this method takes long hours of computation. This result comes from the perfectness check used in many iterations. In this project, we faced the trade-off between the randomness and the time complexity for different methods. Also, in our experimentation sessions, there is an experimental limit for generated graphs in terms of density and size. We could not generate any perfect graph in some specific size and density for the heuristic method in reasonable time.

As a future work, we propose some improvements for the heuristic repair algorithm. The algorithm is needed to be tested with the clean odd cycle-free graphs as starting graphs. Also, the combinatorial algorithm can be improved by adding new possible operations such as amalgam~\cite{burlet}, 2-amalgam~\cite{cornuejols} or 2-join, double star cut-set or skew partition~\cite{trotignon} in addition to the ones defined in this study. These operations are more complex, but it gives an opportunity to generate more diverse graphs. 

\newpage

\printbibliography[heading=bibintoc]

\newpage

\appendix

\section{Pseudocodes} \label{sec:pseudo}

\begin{algorithm}[H]
\SetKwInOut{Input}{Input}
\SetKwInOut{Output}{Output}
\caption{Combinatorial Algorithm}
\label{alg:perfectGen}
\Input{$n_1$, $n_2$: the sizes of two perfect graphs}
\Output{A perfect graph $G$}
\SetAlgoLined
\SetKwFunction{RandomCombinatorialOperation}{RandomCombinatorialOperation}
\SetKwFunction{CliqueIdentification}{CliqueIdentification}
\SetKwFunction{Substitution}{Substitution}
\SetKwFunction{Composition}{Composition}
\SetKwFunction{DisjointUnion}{DisjointUnion}
\SetKwFunction{Join}{Join}
\SetKwFunction{Complement}{Complement}

\BlankLine
$\rho \gets$\ \RandomCombinatorialOperation{} \;
    
    \BlankLine
    Let $I$ be the input graph count.\;
    
    \eIf{$\rho =$ complement}{
        $I \gets 1$\;
    }{
        $I \gets 2$\;
    }
    \BlankLine
    For each integer $i \leq I$, let $G_{i} = (V_i, E_i)$ be a graph selected randomly from the collection of small-sized perfect graphs $P_i$ such that $|V_i| = n_i$\;

    Let $G = (V,E)$ be the resulting matrix.\;
    \BlankLine
    \tcc{Choose the appropriate combinatorial operation.}
    \BlankLine
    \uIf{$\rho =$ clique identification}{
        $G \gets $ \CliqueIdentification{$G_{1}$, $G_{2}$} \;
    }
    \uElseIf{$\rho =$ substitution}{
        $G \gets $ \Substitution{$G_{1}$, $G_{2}$} \;
    }
    \uElseIf{$\rho =$ complement}{
        $G \gets $ \Complement{$G_{1}$, $G_{2}$} \;
    }
    \uElseIf{$\rho =$ disjoint union}{
        $G \gets $ \DisjointUnion{$G_{1}$, $G_{2}$} \;
    }
    \uElseIf{$\rho =$ join}{
        $G \gets $ \Join{$G_{1}$, $G_{2}$} \;
    }
    \ElseIf{$\rho =$ complement}{
        $G \gets $ \Complement{$G_{1}$} \;
    }
    \BlankLine
    \Return{$G = (V,E)$}
\end{algorithm}

\begin{algorithm}[H]
  \SetKwInOut{Input}{Input}
  \SetKwInOut{Output}{Output}
  \caption{Parent Vertex Approach in Heuristic Repair Algorithm}
  \label{alg:parentvertex}
  \Input{Starting Graph $G$}
  \Output{Edge-Modified Graph}
  \SetAlgoLined
  \SetKwFunction{BFS}{BFS}
  \SetKwFunction{RemoveEdge}{RemoveEdge}
  \SetKwFunction{AddEdge}{AddEdge}
  
  \BlankLine
  Select a vertex randomly as the starting vertex\;
  Construct the BFS tree using \BFS{G, starting vertex}\;
  
  \BlankLine
  \For{each vertex $v$ in the graph}{
    \If{$v$ is not the starting vertex}{
      \If{$v$ has an edge with a vertex at the same level in the BFS tree}{
        Remove the edge between $v$ and its parent in the BFS tree using \RemoveEdge{v, parent($v$)}\;
        Select a neighbor $n$ of $v$\;
        Add an edge between $n$ and $v$ using \AddEdge{n, parent($v$)}\;
      }
    }
  }
\end{algorithm}


\begin{algorithm}[H]
  \SetKwInOut{Input}{Input}
  \SetKwInOut{Output}{Output}
  \caption{Starting Vertex Approach in Heuristic Repair Algorithm}
  \label{alg:startingvertex}
  \Input{Starting Graph $G$}
  \Output{Edge-Modified Graph}
  \SetAlgoLined
  \SetKwFunction{BFS}{BFS}
  \SetKwFunction{ShortestPath}{ShortestPath}
  \SetKwFunction{AddEdge}{AddEdge}
  
  \BlankLine
  Select a vertex randomly as the starting vertex\;
  Construct the BFS tree using \BFS{G, starting vertex}\;
  
  \BlankLine
  \For{each vertex $v$ in the graph}{
    Identify whether $v$ has an edge with a vertex at the same level in the BFS tree\;
    \If{an edge exists}{
      Remove the edge between $v$ and its parent in the BFS tree\;
    }
    Determine the shortest path from the starting vertex to $v$ using \ShortestPath{G, starting vertex, $v$}\;
    \If{the shortest path traverses an even number of vertices}{
      \AddEdge{G, $v$, starting vertex}\;
    }
    \ElseIf{the shortest path traverses an odd number of vertices}{
      Select one of $v$'s neighbors from the vertices on the shortest path as $u$\;
      \AddEdge{G, $u$, starting vertex}\;
    }
  }
\end{algorithm}

\begin{algorithm}[H]
  \SetAlgoLined
  \SetKwInOut{Input}{Input}
  \SetKwInOut{Output}{Output}
  \label{alg:c5check}
  \Input{A graph $G$}
  \Output{Whether the graph $G$ is C5-free or not}
  
  \ForEach{edge $e$ in $G$}{
    Let $(u, v)$ be the endpoints of edge $e$\;
    
    Remove edge $e$ from $G$\;
    
    Let $neighbors1$ be the neighbors of $u$ in $G$\;
    Let $neighbors2$ be the neighbors of $v$ in $G$\;
    
    \ForEach{vertex $w$ in $neighbors1$}{
      \If{$w$ is in $neighbors2$}{
        \textbf{Output} "Graph $G$ is not C5-free."\;
        \textbf{Terminate}\;
      }
    }
    
    Add edge $e$ back to $G$\;
  }
  
  \textbf{Output} "Graph $G$ is C5-free."\;
  
  \caption{C5-Free Graph Recognition Algorithm}
\end{algorithm}


\begin{algorithm}[H]
  \SetAlgoLined
  \SetKwInOut{Input}{Input}
  \SetKwInOut{Output}{Output}
  \label{alg:c5gen}
  \Input{Number of vertices $N$, desired density $D$}
  \Output{Generated C5-free graph $G$}
  
  Create an empty graph $G$ with $N$ vertices\;
  
  $EdgeCount \leftarrow D \times N \times (N - 1) / 2$\;
  
  \While{the number of edges in $G$ is less than $EdgeCount$}{
    Let $u$ be a random vertex index from $0$ to $N - 1$\;
    Let $v$ be a random vertex index from $0$ to $N - 1$\;
    
    \If{$u$ and $v$ are the same}{
      continue to the next iteration\;
    }
    
    \If{there is already an edge between $u$ and $v$ in $G$}{
      continue to the next iteration\;
    }
    
    Add an edge between $u$ and $v$ in $G$\;
  }
  
  \textbf{Output} the generated C5-free graph $G$\;
  
  \caption{C5-Free Graph Generation Algorithm}
\end{algorithm}



\begin{algorithm}[H]
  \SetAlgoLined
  \SetKwInOut{Input}{Input}
  \SetKwInOut{Output}{Output}
  \label{alg:cleanrec}
  \caption{Clean Odd Cycle-free Graph Recognition}
  
  \Input{Graph $g$, vertices $v1$ and $v2$}
  \Output{Whether the graph $g$ is clean odd cycle-free or not}
  
  \SetKwFunction{CleanCycleCheck}{clean\_cycle\_check}
  \SetKwProg{Fn}{Function}{:}{}
  
  \Fn{\CleanCycleCheck{$g$, $v1$, $v2$}}{
    Create a list $vlist$ containing all vertices in $g$ except $v1$ and $v2$\;
    
    \ForEach{vertex $v3$ in $vlist$}{
      Calculate the shortest paths $p13$ from $v1$ to $v3$ and $p23$ from $v2$ to $v3$\;
      
      \If{the union of $p13$ and $p23$ contains at least 5 distinct vertices and has an odd length}{
        Create a subgraph $indg$ of $g$ with the vertices in the union of $p13$ and $p23$\;
        
        \If{all vertices in $indg$ have degree 2}{
          \Return \textbf{true} (Graph $g$ is clean odd cycle-free)\;
        }
      }
    }
    
    \Return \textbf{false} (Graph $g$ is not clean odd cycle-free)\;
  }
\end{algorithm}


\begin{algorithm}[H]
  \SetAlgoLined
  \SetKwInOut{Input}{Input}
  \SetKwInOut{Output}{Output}
  
  \caption{Clean Odd Cycle-free Graph Generation (ER Model)}
  \label{alg:cleangen}
  \Input{Number of vertices $n$, edge probability $p$}
  \Output{Generated clean odd cycle-free graph $g$}
  
  \SetKwFunction{CleanCycleCheck}{clean\_cycle\_check}
  \SetKwProg{Fn}{Function}{:}{}
  \SetKwRepeat{Do}{do}{while}
  \SetKwFor{For}{for}{do}{endfor}
  
  \Fn{CleanCycleGenerationER($n$, $p$)}{
    Create an empty graph $g$ with $n$ vertices\;
    Calculate the number of edges $e$ based on $n$ and $p$\;
    
    \Do{the number of edges in $g$ is less than $e$}{
      Generate a random number of edges $number\_of\_edge$ using a Poisson distribution with mean 0.25\;
      
      \For{$i$ in $1$ to $number\_of\_edge$}{
        Select two random vertices $v1$ and $v2$ from $g$\;
        Add an edge between $v1$ and $v2$ in $g$\;
      }
      
      Create an empty list $edge\_result$\;
      
      \For{$i$ in $1$ to $number\_of\_edge$}{
        $edge\_result[i] \leftarrow$ \CleanCycleCheck{$g$, $v1$, $v2$}\;
      }
      
      \If{any edge in $edge\_result$ allows a clean cycle}{
        \For{$i$ in $1$ to $number\_of\_edge$}{
          Remove the edge between $v1$ and $v2$ from $g$\;
        }
      }
      
      Simplify $g$ by removing duplicate edges and self-loops\;
    }
    
    \Return $g$\;
  }
\end{algorithm}


\end{document}