\section{Introduction}

\subsection{Summary}

The study of ``perfect graphs'' has been an intriguing and challenging area in graph theory, with numerous applications in fields such as optimization, network design, and scheduling. These graphs possess remarkable structural properties and have been extensively studied over the years.

The problem addressed in this study is the generation of random perfect graphs, which plays a crucial role in understanding their properties and developing algorithms for practical applications. The aim of this study is to investigate and analyze various algorithms for generating random perfect graphs. In this context, the primary objective is to develop efficient and effective methods for generating random perfect graphs that accurately capture the key properties of these graphs. Additionally, we seek to compare the algorithms by exploring the statistical properties and structural characteristics of the generated graphs, providing insights into the behavior and distribution of random perfect graphs.

The findings of this research will not only deepen our understanding of perfect graphs but also have practical implications for algorithm design and problem-solving on perfect graphs in real-world applications.

\subsection{Short Overview of Identification, Analysis and Solution Methodologies}


Firstly, the problem is clearly defined and the study's objectives are outlined, considering the necessary requirements and limitations. Fundamental concepts and definitions in perfect graph theory, such as perfectness, cliques, and related properties are also reviewed. Then, existing methods for generating perfect graphs in the literature such as Erdős and Rényi's approach~\cite{erdos} are examined. Building upon this foundation, novel algorithms for generating random perfect graphs like ``heuristic repair algorithm'' and ``combinatorial algorithm'' are proposed and their performance in terms of efficiency and quality of the generated graphs are analyzed. Further, an extensive experimental evaluation is conducted to validate the proposed algorithms and assess the statistical properties of the generated random perfect graphs. The generated graphs are compared against known benchmarks and their performance are assessed using various graph metrics and properties to provide valuable insights into the effectiveness and limitations of the proposed algorithms. Finally, the work is concluded by summarizing the key findings and discussing the potential future improvements to the algorithms.

\newpage

\tableofcontents