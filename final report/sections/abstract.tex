\begin{abstract}
    This project focuses on the generation of random perfect graphs, specifically exploring their practicality for testing algorithms designed for perfect graphs. Different algorithms are developed and compared through a detailed analysis of their performance, namely the ``heuristic repair method'' and the ``combinatorial approach''. The study aims to generate graphs that have the properties of perfect graphs and evaluate the efficiency and accuracy of each algorithm. The outcomes of this research contribute to algorithm testing and performance evaluation, aiding researchers in selecting suitable methods for their requirements and advancing the field of perfect graph theory. In result, the combinatorial algorithm works faster, but it is limited by a number of operations; whereas, the heuristic repair approach has the possibility to generate every perfect graph with a positive probability even though it takes much more time compared to the combinatorial method.
\end{abstract}

\begin{otherlanguage}{turkish}

    \begin{abstract}
        Bu proje, özellikle ``mükemmel'' çizgeler için tasarlanmış algoritmaların test edilmesi açısından rastgele mükemmel çizgelerin üretimine odaklanmaktadır. ``Sezgisel onarım yöntemi'' ve ``kombinatoryal yaklaşım'' gibi farklı algoritmalar geliştirilmiş ve performanslarının ayrıntılı bir analizi yoluyla karşılaştırılmıştır. Çalışmanın amacı, mükemmel çizgelerin özelliklerine sahip çizgelerin üretilmesini sağlamak ve her bir algoritmanın etkinlik ve doğruluklarını değerlendirmektir. Araştırmanın sonuçları, algoritma testi ve performans değerlendirmesine katkıda bulunarak araştırmacıların ihtiyaçlarına uygun yöntemleri seçmelerine yardımcı olacak ve mükemmel çizge kuramı alanının gelişmesine katkı sağlayacaktır. Sonuç olarak, kombinatoryal algoritma daha hızlı ancak sayılı sayıda işlemle sınırlı kalmaktadır; sezgisel yaklaşım ise, kombinatoryal yönteme göre çok daha fazla zaman almasına rağmen her mükemmel çizgeyi pozitif bir olasılıkla oluşturma olanağına sahiptir.
    \end{abstract}

\end{otherlanguage}